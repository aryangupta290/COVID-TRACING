\documentclass{article}
\usepackage{graphicx}
\begin{document}
\begin{titlepage}

\centering
\vfill
\includegraphics{iiit}
\vskip1cm

{\bfseries\Large
		International Institute of Information Technology, Hyderabad\\
        \vskip1cm
        Data Structures and Algorithms \
        \vskip1cm
        S21CS1.201\
        \vskip2cm
        COVID-Tracing Mini-Project\\
        \vskip2cm
       Team 6}\\
        \vskip 1cm
{\bfseries        
        Vayur S (2020112027)\\
        \vskip 0.5cm
        Khushi Agarwal (2020101092)\\
        \vskip 0.5cm
        Mancharla Harish (2020102062)\\
        \vskip 0.5cm
        Abhinav Tanniru (2020112007)\\
        \vskip 0.5cm
        Aryan Gupta (2020101091)}
    


\end{titlepage}
\newpage

\section{Data Structures Used:}
The project contains multiple structures for storing
\begin{enumerate}
    \item Routes: A linked list for storing the routes in a given graph. 
    \item Person: It contains the status of a person (if the person is covid-positive or a contact etc.), length of his/her infection, who was it's cause and the station number.
    \item Stations: It has a list of all the people currently in the station and it's current danger value.
    \item Paths: A linked list for storing the routes of multiple people, it contains the person id, the route and a pointer to the next path structure.
    \item Days: An array of structures which consists of stations, paths and people and can store these for up to 15 days.
    \item Possible Paths: It is used to store the 3(or less) possible routes a person can take and their corresponding danger values.
\end{enumerate}

The graph implementation is done using adjacency lists and min-heaps.


\newpage
\section{Algorithms Used:}


\newpage
\section{Division Of Work:}
\begin{itemize}
    \item Aryan: 3-way Dijkstra algorithm

    \item Khushi: 3-way Dijkstra algorithm

    \item Vayur: Update functions and structs

\item Abhinav: Update functions and structs

\item Harish: main.c
\end{itemize}

\end{document}
